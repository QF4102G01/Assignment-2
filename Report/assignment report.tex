% This is a comment.
% the region directly below this comment, up till the command \begin{document} is known as the 'preamble'
% basic setup
\documentclass{article}
\usepackage[english]{babel}
\usepackage[utf8]{inputenc}

% for mathematics
\usepackage{amsmath}
\usepackage{amsthm}
% define theorems, lemmas, etc
\newtheorem{theorem}{Theorem}
\newtheorem{lemma}{Lemma}
\newtheorem{corollary}{Corollary}
\newtheorem{definition}{Definition}
\newtheorem{example}{Example}
\usepackage{amssymb}

% for adjusting margins
\usepackage{geometry}
\geometry{
	a4paper,
 	left=26mm,
 	right=20mm,
 	top=33mm,
 	bottom=38mm
}

% for introducing urls
\usepackage{url}

% for colored text
\usepackage{color}

% for creating lists
\usepackage{enumerate}

% for import graphics
\usepackage{graphicx}

% include algorithm package
\usepackage[]{algorithm2e}

% change font to times new roman
%\usepackage{times}

% add padding to in between paragraphs
\setlength{\parskip}{1em}

% eliminate indent at start of paragraph
\setlength\parindent{0pt}

% title details
\title{QF4102 Financial Modelling and Computation Assignment 2}
%\date{}
\author{G01 Wang Zexin, Chen Penghao}

%~~~~~~~~~~~~~~~~~~~~~~~~~~~~~~~~~~~~~~~~~~~~~~~~~~~~~~~~~~~~~~~~~~~~~~~~~~~~~~
\begin{document}

% insert title
\maketitle
% make a new page
\newpage

\section{Forward Grid Shooting Method on American floating-strike arithmetic-average call}
\subsection*{\emph{Statement of the problem}}
Write a MatLab function to calculate the American fixed-strike arithmetic-average put option value using Forward Shooting Grid Method with linear interpolation.

The American fixed-strike arithmetic-average put option is initiated 0.25 year ago, and has 0.25 year more to expiry. The underlier price is currently $\$100$, volatility $40%$ and dividend yield 0.01 with a running average of $\$95$, which was taken over the earlier 0.25 year since launched. The risk-free rate $r_f$ is $0.1$ and strike price $K=
\$100$.

Compute the option value with $\rho=1$, $\rho=\dfrac{1}{2}$ and $\rho = \dfrac{1}{5}$. For each $\rho$, the number of periods in the lattice is required to be 50, 100, 200 and 400. Compare the computed results and the computation times taken.

Next, an American fixed-strike lookback put option is concerned, initiated 0.25 year ago, and still having 0.25 year to expiry. The current underlier price $S_{0.25}=\$1$, volatility $\sigma=40\%$ and dividend yield $q = 0.01$. The running minimum is at $\$0.97$ which was taken over the earlier period of 0.25 year. The risk-free rate $r_f=0.1$ and strike $K=\$0.95$.

Compute the function for number of time periods $N = 50$ to $N = 500$ with increments of 50, and comment on the numerical results and computation time taken. Lastly, re-run the algorithm with running minimum changed from $\$0.97$ to $\$0.57$.

\subsection{Description of work done}
First we implemented the FSGM for American fixed-strike arithmetic-average put option.

\begin{algorithm}[H]
	\KwData{runningTime, $\tau$, $S_0$, $\sigma$, $q$, $A$, $r$, $K$, $N$, $\rho$}
	\KwResult{$P_{FSGM}$, Option Premium}
	$\Delta t$ = $\dfrac{\tau}{N}$;
	$\Delta x = \sigma \sqrt{\Delta t}$;
	$\Delta y = \rho \Delta x$\;
	$u = e^{\Delta x}$;
	$d = e^{-\Delta x}$;
	$p = \dfrac{e^{r \Delta t} - d}{u - d}$\;
	
\caption{Algorithm for pricing European down-and-out option}
\end{algorithm}

We

\subsection{Comment on the numerical results and computation times taken for different $\rho$ and $N$ values}
The different amoung of time taken and option prices for various set of $\rho$ and $N$ is shown below:

\hspace{23mm} When $\rho=1$:
\begin{center}
	\begin{tabular}{| c | c | c | c |}
		\hline $\rho$ & $N$ & Option price & Time taken in $\mu s$\\
		[0.5ex]
		\hline 1 & 50 & 5.3793 & 0.26287 \\
		\hline 1 & 100 & 5.4178 & 2.649 \\
		\hline 1 & 200 & 5.4428 & 16.86 \\
		\hline 1 & 400 & 5.4553 & 139.52 \\
		\hline
	\end{tabular}
\end{center}

\hspace{23mm} When $\rho=0.5$:
\begin{center}
	\begin{tabular}{| c | c | c | c |}
		\hline $\rho$ & $N$ & Option price & Time taken in $\mu s$\\
		[0.5ex]
		\hline 0.5 & 50 & 5.3456 & 0.60326 \\
		\hline 0.5 & 100 & 5.4058 & 3.9802 \\
		\hline 0.5 & 200 & 5.4361 & 31.521 \\
		\hline 0.5 & 400 & 5.4509 & 299.12 \\
		\hline
	\end{tabular}
\end{center}

\hspace{23mm} When $\rho=0.2$:
\begin{center}
	\begin{tabular}{| c | c | c | c |}
		\hline $\rho$ & $N$ & Option price & Time taken in $\mu s$\\
		[0.5ex]
		\hline 0.2 & 50 & 5.3391 & 2.1132 \\
		\hline 0.2 & 100 & 5.3992 & 12.573 \\
		\hline 0.2 & 200 & 5.4318 & 92.8 \\
		\hline 0.2 & 400 & 5.4487 & 760.81 \\
		\hline
	\end{tabular}
\end{center}

Regarding the option value, using the same $\rho$ value, as $N$ increases, the estimated option value increases from $N = 50$ to $N = 400$. 

\subsection{Comment on the numerical results and computation times taken for different $N$ values}

\subsection{Comment on the numerical results obtained from lower running minimum value}

\section{Explicit Difference Scheme III for vanilla call option}
\subsection*{\emph{Statement of the problem}}
Write a MatLab function to implement the explicit difference scheme III for pricing European vanilla calls. Test the algorithm out with a European vanilla call option with strike $K = \$9$, time to maturity $0.25$ year, current underlier price $S = \$9.8$, volatility $\sigma=0.15$. The dividend yield $q = 1\%$ and risk-free rate is $0.1\%$.

\subsection{Newly issued European floating strike lookback put options}

\subsection{Previously issued European floating strike lookback put options}

\subsection{Analyze, compare and comment on the results}

\end{document}